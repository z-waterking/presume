\documentclass[UTF8, 11pt, fontset=overleaf]{ctexart}

% Standard margins for better readability and filling more pages
\usepackage[a4paper, margin=0.7in]{geometry}
\usepackage{titlesec}
\usepackage{array}

% Increase line spacing
\linespread{1.15}
% Add paragraph spacing
\setlength{\parskip}{0.3em}

% Disable page numbers
\pagestyle{empty}

% --- Section Title Styling ---
\titleformat{\section}{\large\heiti\bfseries}{}{0em}{}[\titlerule]
% Increased section spacing: {left}{before}{after}
\titlespacing{\section}{0pt}{12pt}{6pt}

\begin{document}
% Removed negative vspace


% --- 个人信息 ---
\begin{center}
    {\huge \heiti \textbf{张思凡}} \\
    \vspace{8pt}
    \small 13260151230 \textbar \ zhangsifanbj@163.com \textbar \ 北京市海淀区
\end{center}

% --- 工作经历 ---
\section*{工作经历}
\noindent
\begin{tabular*}{\textwidth}{@{\extracolsep{\fill}} l l r}
    \textbf{微软互联网工程院} & SDE 2 & 2022.11-至今 \\
    \textbf{阿里巴巴 AE 技术部} & 推荐算法工程师 & 2020.07-2022.10 \\
    \textbf{阿里巴巴 \& 亚马逊AWS} & 算法工程师(实习) & 2019.05-2019.09 \\
\end{tabular*}

% --- 项目经历 ---
\section*{项目经历}

% --- 微软项目 ---
\noindent \textbf{微软互联网工程院} \hfill \textbf{2022.11-至今} \\
\noindent \textbf{分别负责 Meta Smart Match (搜索广告匹配) 及 BingViz Agents (Agent 平台) 的核心研发,在搜推算法与 AI Agent 领域取得显著业务落地成果。}

\vspace{3pt}

% MSM
\noindent $\bullet$ \textbf{搜索广告: 针对长尾流量变现效率低痛点,主导智能语义匹配系统 (MSM) 的召回与排序升级} \\
\noindent \textbf{语义召回体系} \hspace{0.3em} 构建 Query-Ad 点击二部图,利用 Transitive Join 挖掘非直观共现关系,并结合离线生成式语义扩展,实现对长尾 Query 意图的精准捕获。 \\
\noindent \textbf{分层排序架构} \hspace{0.3em} 设计 L1 粗排 + L2 精排的漏斗架构。在 L1 阶段引入 Pareto Optimization 思想,通过多目标公式融合 Rank Score(语义相关性)、CTR 预估与 Bid,在保证用户体验的前提下最大化商业变现效率。北美市场 RPM +1.8\%,CN 市场 Revenue +2\%。\textbf{获 Greatness Award。}

\vspace{3pt}

% Agent & SkillLoop
\noindent $\bullet$ \textbf{BingViz Agent: 构建深度业务分析框架(DAF), 并实践“专家引导+自主探索”的 OPE 开发范式} \\
\noindent \textbf{深度思考引擎} \hspace{0.3em} 主导开发基于 LLM + MCP (Model Context Protocol) 的 Agent 框架。设计 Workflow-based 范式,取代不稳定的 ReAct 循环, 实现“大盘扫描 $\rightarrow$ 维度拆解 $\rightarrow$ 竞品对比”的长链路思维链执行。 \\
\noindent \textbf{范式验证与落地} \hspace{0.3em} 提出“数据完备+专家经验”的开放式分析范式: 为 Agent 提供充足业务数据与专家先验上下文, 允许其自主探索分析路径与归因逻辑。基于此能力, 1) 自动归因欧洲市场 RPM 波动, 直接支持业务决策; 2) 独立构建 SkillLoop 项目, 解决企业隐性知识交换难题, 通过深度挖掘用户学习意图实现精准学习推荐, \textbf{荣获 Global Hackathon 全球三等奖}, 并代表中国区向 Microsoft AI CEO (Mustafa Suleyman) 汇报演示。

\vspace{12pt}
\noindent \rule{\textwidth}{0.5pt} % 分隔符
\vspace{12pt}
\noindent \textbf{阿里巴巴 AE 技术部} \hfill \textbf{2020.07-2022.10} \\
\noindent \textbf{分别负责 AE 推荐业务中排序及召回模块,从全链路视角进行迭代升级,不断提升整个推荐链路的业务效率,助力系统精准匹配用户兴趣,取得了加收转化率+7\%,UV价值+12\%的效果。}

\vspace{3pt}

% 排序模块
\noindent $\bullet$ \textbf{排序: 针对推荐策略实时调控的需要,提出实时深度控制模块,并横向拓展助力全场域提升} \\
\noindent \textbf{深度控制 LTR} \hspace{0.3em} 作为大模型的后链路,摒弃传统的手工调权,直接设定线上期望业务目标,通过深度控制网络(PID),动态调整各 label 所对应的loss融合权重,以寻找当前限制下的最优参数。通过控制离线 auc及模型参数变化的阈值,来保证在线效果的稳定性。取得UV价值+12\%的效果。 \\
\noindent \textbf{全场域优化} \hspace{0.3em} 对实时流程的上下游进行了抽象。样本侧,设置缓存区以归因延迟label,设置缓冲队列以保持正负样本比例稳定。利用drop rank 选取大模型 top10\%的特征用于LTR训练。构建了从数据源、样本构造、到模型训练、线上生效等一整套配置化流程。于各个场域均取得了IPV+2\%,UV价值+2\%以上的收益。

\vspace{3pt}

% 召回模块
\noindent $\bullet$ \textbf{召回: 针对多路召回合并时带来的马太效应,提出个性化多路召回融合模块} \\
\noindent \textbf{个性化召回融合} \hspace{0.3em} 针对线上多路召回(>=10)合并时,截断数量由人工拍定造成的马太效应与效率损失问题,提出了个性化召回融合方法。利用黑盒优化的思想,引入投票理论和borda 计数法,根据item的相对顺序及召回对应的权重,确定item 最终顺序。离线侧,采用生成器动态生成各路召回融合权重;在线侧,利用评估器对不同权重的线上效果进行评估,同时指导生成器的训练。取得了GMV+2.4\%的效果。

\vspace{3pt}

% 全链路模块
\noindent $\bullet$ \textbf{全链路: 离线评估召回有效性,研究召回与排序的漏斗模式带来的耦合问题} \\
\noindent \textbf{离线评估} \hspace{0.3em} 搭建hitrate 评估流程,通过对超过20种召回源进行统一尺度下的评估,以离线验证召回数据源的有效性;通过各路召回组合评估,以发现可互补的召回源;通过粗排打分结果评估,以评定粗排的有效性。 \\
\noindent \textbf{解除耦合} \hspace{0.3em} 针对推荐系统漏斗模式所带来的上下游链路耦合问题,评估各路召回曝光占比及转化效率,以验证耦合现象的存在。与精排联动,对召回来源进行泛化,削减耦合程度,助力更多召回源上线。

% --- 发表论文 ---
\section*{发表论文}
\noindent 1. Yifan Zhu, Sifan Zhang, etc. Social weather: A review of crowdsourcing-assisted meteorological knowledge services through social cyberspace[J]. Geoscience Data Journal, 2020 (SCI-2区) \\
\noindent 2. Yubing Nie, Yifan Zhu, Sifan Zhang, etc. Academic rising star prediction via scholar's evaluation model and machine learning techniques[J]. Scientometrics, 2019. (SCI-2区)

% --- 教育背景 ---
\section*{教育背景}
\noindent
\begin{tabular*}{\textwidth}{@{\extracolsep{\fill}} l l l r}
    \textbf{北京理工大学(985)} & 硕士 & 计算机学院 计算机科学与技术 & 2017.09-2020.07 \\
    \textbf{暨南大学(211)} & 本科 & 电气信息学院 包装工程 & 2012.09-2016.07 \\
\end{tabular*}

% --- 获奖情况 ---
\section*{获奖情况}
\noindent Microsoft Global Hackathon 2025 \textbf{Global 3rd Place (全球季军)} \hfill 2025.09 \\
\noindent Microsoft Greatness Award \hfill 2024.12 \\
\noindent 北京市2020年优秀毕业生 \hfill 2020.07 \\
\noindent 北京大学生优秀创业团队二等奖,创立了智法互动创业团队,已投入运营 \hfill 2019.09 \\
\noindent 工信创新奖学金 \hfill 2019.04

\end{document}