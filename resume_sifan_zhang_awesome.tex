\documentclass[UTF8, 11pt, fontset=overleaf]{ctexart}

% 页面布局优化:进一步缩小边距以确保一页放下
\usepackage[a4paper, top=0.4in, bottom=0.4in, left=0.5in, right=0.5in]{geometry}
\usepackage{titlesec}
\usepackage{enumitem}
\usepackage{array}
\usepackage{amssymb} % 用于方块符号

% 禁用页码
\pagestyle{empty}

% --- 标题样式设置 ---
% 使用黑体 (\heiti) 作为标题
\titleformat{\section}{\large\heiti\bfseries}{}{0em}{}[\titlerule]
\titlespacing{\section}{0pt}{6pt}{4pt} % 极度压缩标题前后间距

\begin{document}
\vspace*{-2.5em} % 整体内容上移

% --- 个人信息 ---
\begin{center}
    {\huge \heiti \textbf{张思凡}} \\
    \vspace{3pt}
    \small 13260151230 \textbar \ zhangsifanbj@163.com \textbar \ 北京市海淀区
\end{center}

% --- 工作经历 ---
\section*{工作经历}
\noindent
\begin{tabular}{@{}p{0.4\textwidth} p{0.35\textwidth} r@{}}
    \textbf{微软互联网工程院} & SDE 2 & 2022.11-至今 \\
    \textbf{阿里巴巴 AE 技术部} & 推荐算法工程师 & 2020.07-2022.10 \\
    \textbf{阿里巴巴 \& 亚马逊AWS} & 算法工程师(实习) & 2019.05-2019.09 \\
\end{tabular}

% --- 项目经历 ---
\section*{项目经历}
\noindent \textbf{阿里巴巴 AE 技术部} \hfill \textbf{2020.07-2022.10} \\
\noindent 分别负责 AE 推荐业务中排序及召回模块,从全链路视角进行迭代升级,不断提升整个推荐链路的业务效率,助力系统精准匹配用户兴趣,取得了加收转化率+7\%,UV价值+12\%的效果。

\vspace{4pt}
% 使用小方块代替 >,如需完全不要符号,可将 label={\tiny$\blacksquare$} 改为 label={}
\begin{itemize}[leftmargin=1.2em, itemsep=2pt, topsep=0pt, label={\tiny$\blacksquare$}]
    \item \textbf{排序: 针对推荐策略实时调控的需要,提出实时深度控制模块,并横向拓展助力全场域提升}
    
    \noindent \textbf{深度控制 LTR} \hspace{0.6em} 作为大模型的后链路,摒弃传统的手工调权,直接设定线上期望业务目标,通过深度控制网络(PID),动态调整各 label 所对应的loss融合权重,以寻找当前限制下的最优参数。通过控制离线 auc及模型参数变化的阈值,来保证在线效果的稳定性。取得UV价值+12\%的效果。
    
    \noindent \textbf{全场域优化} \hspace{0.6em} 对实时流程的上下游进行了抽象。样本侧,设置缓存区以归因延迟label,设置缓冲队列以保持正负样本比例稳定。利用drop rank 选取大模型 top10\%的特征用于LTR训练。构建了从数据源、样本构造、到模型训练、线上生效等一整套配置化流程。根据不同场景的业务目标及场域特性,分别选用相应的特征、模型及打分组合公式,并进行超参寻优。于各个场域均取得了IPV+2\%,UV价值+2\%以上的收益。

    \item \textbf{召回: 针对多路召回合并时带来的马太效应,提出个性化多路召回融合模块}
    
    \noindent \textbf{个性化召回融合} \hspace{0.6em} 针对线上多路召回(>=10)合并时,截断数量由人工拍定造成的马太效应与效率损失问题,提出了个性化召回融合方法。利用黑盒优化的思想,引入投票理论和borda 计数法,根据item的相对顺序及召回对应的权重,确定item 最终顺序。离线侧,采用生成器动态生成各路召回融合权重;在线侧,利用评估器对不同权重的线上效果进行评估,同时指导生成器的训练。取得了GMV+2.4\%的效果。

    \item \textbf{全链路: 离线评估召回有效性,研究召回与排序的漏斗模式带来的耦合问题}
    
    \noindent \textbf{离线评估} \hspace{0.6em} 搭建hitrate 评估流程,通过对超过20种召回源进行统一尺度下的评估,以离线验证召回数据源的有效性;通过各路召回组合评估,以发现可互补的召回源;通过粗排打分结果评估,以评定粗排的有效性。
    
    \noindent \textbf{解除耦合} \hspace{0.6em} 针对推荐系统漏斗模式所带来的上下游链路耦合问题,评估各路召回曝光占比及转化效率,以验证耦合现象的存在。与精排联动,对召回来源进行泛化,削减耦合程度,助力更多召回源上线。
\end{itemize}

% --- 发表论文 ---
\section*{发表论文}
\noindent 1. Yifan Zhu, Sifan Zhang, etc. Social weather: A review of crowdsourcing-assisted meteorological knowledge services through social cyberspace[J]. Geoscience Data Journal, 2020 (SCI-2区) \\
\noindent 2. Yubing Nie, Yifan Zhu, Sifan Zhang, etc. Academic rising star prediction via scholar's evaluation model and machine learning techniques[J]. Scientometrics, 2019. (SCI-2区)

% --- 教育背景 ---
\section*{教育背景}
\noindent
\begin{tabular}{@{}l l l r@{}}
    \textbf{北京理工大学(985)} & 硕士 & 计算机学院 计算机科学与技术 & 2017.09-2020.07 \\
    \textbf{暨南大学(211)} & 本科 & 电气信息学院 包装工程 & 2012.09-2016.07 \\
\end{tabular}

% --- 获奖情况 ---
\section*{获奖情况}
\begin{itemize}[leftmargin=0em, itemsep=0pt, topsep=2pt, parsep=0pt, label={}]
    \item 北京市2020年优秀毕业生 \hfill 2020.07
    \item 北京大学生优秀创业团队二等奖,创立了智法互动创业团队,已投入运营 \hfill 2019.09
    \item 工信创新奖学金 \hfill 2019.04
\end{itemize}

\end{document}